\documentclass[10pt]{book}
\usepackage[utf8]{inputenc}

\title{THP Code of Conduct}
\author{\texttt{\#00000000 (IWant2PlayTouhou/bluebear94)}}
\date{July 2014}

\usepackage{natbib}
\usepackage{graphicx}

\begin{document}

\maketitle

\tableofcontents

\section{Introduction}

On January 26, 2014, Touhou Prono was born. But for five months, it fell asleep. And we certainly cannot have an anarchy.

\chapter{Rules for contributing}

\section{Purpose}

The purpose of this chapter is to maintain consistency in the text.

\section{Use of RFC 2119}

The terms MUST, etc.~are defined in RFC 2119 (\texttt{http://www.ietf.org/rfc/rfc2119.txt}), and in this text, may be lowercased for convenience.

\section{Numbers}

Excluding chapter, section, and subsection numbers, plus numbers starting an entry in a numbered list and dates, numbers must be spelled out first then its value in numerals must be given in parentheses.

Ordinal numbers must not have a superscripted part.

\section{Enumerate or itemize?}

All lists must be numbered rather than bulleted. There should be only one (1) list per subsection, or (incl) if none, section.

\section{Contractions}

Contractions must not be used.

\section{Quotation marks}

Punctuation must fall outside the quotation mark unless it is part of the quote.

\section{Possessives}

The ``'s'' form must be used for animate objects of two (2) words or fewer, except if there are two (2) in a row (``A's B's C'' is not permitted; ``C of A's B'' is the required form); otherwise, the ``of'' form must be used.

\section{``Or''}

The ambiguous term ``or'' must not be used alone; instead, it shall be used in one of the following contexts:

\begin{enumerate}
 \item ``Or (incl)'' is defined as at least one of the two or more items joined.
 \item ``Or (excl)'' is defined as only one of the exactly two items joined.
\end{enumerate}

``Or (incl)'' or simply ``or'' (see below) should be used when neither of the above phrases changes the meaning.

The sole exceptions are:
\begin{enumerate}
 \item The phrase ``he or she'' and similar forms as a gender-neutral pronoun
 \item X or [more or (excl) fewer]
 \item Other cases when two clauses are mutually exclusive
\end{enumerate}

\section{Hierarchy}

A chapter or section may contain lower points (sections and subsections, respectfully) if and only if there is no text before the first (1st) lower point. A chapter should contain sections, but a section may choose to have no subsections.

\section{Writing dates}

Dates in the Gregorian calendar must be formatted with the year, then the month spelled out, then the day.

\chapter{Officiality of the Code}

Effective 2014 July 13, only the \texttt{master} branch of the \texttt{touhou-prono/thp-code} repository on the GitHub web site possesses legal value.

\chapter{Structure of Government}

\section{Members}

\subsection{Definition}

Members are people who have joined the \texttt{touhouprono} Google group.

\subsection{GitHub}

If a Member does not have an account on the GitHub web site, then he or she shall be removed from membership.

\subsection{Identification}

Each Member shall be assigned an unsigned thirty-two (32) -bit identifier prefixed with an octothrope (\texttt{\#}), and a table mapping from these identifiers to their chosen pseudonyms shall be made public.

\section{The Leader}

\subsection{Powers}

The Leader has the power to:

\begin{enumerate}
 \item Arbitrarily modify the Code, primarily to codify basic mechanics
 \item Admit or (incl) remove any Member to or (incl) from the group
 \item Establish an official language
 \item Not perform one or more of these actions
\end{enumerate}

These powers should not be used without reason. In particular, it is not recommended to change fundamental text in the Code.

\subsection{Responsibilities}

The Leader is liable to:

\begin{enumerate}
 \item Use his or her powers in order to enhance Touhou Prono
 \item Enforce the Code fairly, including on him- or herself
 \item If establishing an official language, using a language widely understood or (incl) easily and quickly learnable
\end{enumerate}

\subsection{Resignation of Leader}

\subsection{Planned resignation}

If a Leader intends to resign, then he or she must notify all Members at least sixteen (16) days in advance. During the first eight (8) days of this period, Members may apply to become the next Leader, and the current Leader must inspect at least the first sixteen (16) applications submitted, or if fewer than sixteen (16) are submitted, all that were, and choose the candidate from these applications.

\subsection{Unplanned resignation}

If an unplanned event causes a Leader to lose his or her ability to fulfill his or her position, a vacancy of sixteen (16) days is declared. During the first eight (8) days of the period, Members may apply to become the next Leader, and any Member who did not apply may vote for one or more candidates (given their applications) during the following eight (8) days, in which case the candidate receiving the most votes becomes the Leader.

\end{document}
