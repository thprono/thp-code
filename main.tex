\documentclass[10pt]{book}
\usepackage[utf8]{inputenc}
\usepackage{amsmath, amssymb}

\title{THP Code of Conduct}
%\author{\texttt{\#00000000 (IWant2PlayTouhou/bluebear94)}}
\date{July 2014}

\usepackage{natbib}
\usepackage{graphicx}
\usepackage{verbatim}

\begin{document}

\maketitle

\tableofcontents

\section{Introduction}

On January 26, 2014, Touhou Prono was born. But for five months, it fell asleep. And we certainly cannot have an anarchy. In order to remove anarchy, we must establish rules.

There exists an adage saying: ``Ignorance of the law is no excuse.'' Yet this saying cannot be applied unless the subjects can access and understand the law. All Members are therefore advised to read this document carefully and know what they can or cannot do in the scope of Touhou Prono.

One may notice that hostility is explicitly permitted in posting. This bit of information has been added because \emph{Touhou Prono is not Omnimaga}. In fact, \texttt{IWant2PlayTouhou} started Touhou Prono after being (temporarily) banned from Omnimaga. We must take into account the values that started Touhou Prono.

\section{TODO}

\begin{enumerate}
 \item Finish the Introduction
\end{enumerate}

\chapter{Rules for contributing}

\section{Purpose}

The purpose of this chapter is to maintain consistency in the text.

\section{Use of RFC 2119}

The terms MUST, etc.~are defined in RFC 2119 (\texttt{http://www.ietf.org/rfc/rfc2119.txt}), and in this text, may be lowercased for convenience.

\section{Numbers}

Excluding chapter, section, and subsection numbers, plus numbers starting an entry in a numbered list and dates, numbers must be spelled out first then its value in numerals must be given in parentheses.

Ordinal numbers must not have a superscripted part.

\section{Enumerate or itemize?}

All lists must be numbered rather than bulleted. There should be only one (1) list per subsection, or (incl) if none, section.

\section{Contractions}

Contractions must not be used.

\section{Quotation marks}

Punctuation must fall outside the quotation mark unless it is part of the quote.

\section{Possessives}

The ``'s'' form must be used for animate objects of two (2) words or fewer, except if there are two (2) in a row (``A's B's C'' is not permitted; ``C of A's B'' is the required form); otherwise, the ``of'' form must be used.

\section{``Or''}

The ambiguous term ``or'' must not be used alone; instead, it shall be used in one of the following contexts:

\begin{enumerate}
 \item ``Or (incl)'' is defined as at least one of the two or more items joined.
 \item ``Or (excl)'' is defined as only one of the exactly two items joined.
\end{enumerate}

``Or (incl)'' or simply ``or'' (see below) should be used when neither of the above phrases changes the meaning.

The sole exceptions are:
\begin{enumerate}
 \item The phrase ``he or she'' and similar forms as a gender-neutral pronoun
 \item X or [more or (excl) fewer]
 \item Other cases when two clauses are mutually exclusive
\end{enumerate}

\section{Hierarchy}

A chapter or section may contain lower points (sections and subsections, respectfully) if and only if there is no text before the first (1st) lower point. A chapter should contain sections, but a section may choose to have no subsections.

\section{Writing dates}

Dates in the Gregorian calendar must be formatted with the year, then the month spelled out, then the day.

\chapter{Officiality of the Code}

\section{Sources of the Code}

Effective 2014 July 13, only the \texttt{master} branch of the \texttt{touhou-prono/thp-code} repository on the GitHub web site possesses legal value.

\section{Comments in \LaTeX\ source}

Comments in the \LaTeX\ source have no legal value.

\section{Scope of enforcement}

The Code has no legal value outside of the domains of Touhou Prono, which include:

\begin{enumerate}
 \item The \texttt{touhouprono} Google Group
 \item The \texttt{touhou-prono} GitHub group, its repositories, and its web pages
\end{enumerate}

\section{Official Language}

The official language of Touhou Prono is English.

The Code has legal value only in the official language.

\section{Dates}

Dates are assumed to be in the time zone of the Leader's residence (see section \ref{sec:leader} for definition of Leader).

\chapter{Structure of Government}

\section{Members}

\subsection{Definition}

Members are people who have joined the \texttt{touhouprono} Google group. The term must be capitalized in order to distinguish them from members of a committee.

\subsection{GitHub}

If a Member does not have an account on the GitHub web site, then he or she shall be removed from membership.

\subsection{Identification}

Each Member shall be assigned an unsigned thirty-two (32) -bit identifier prefixed with an octothrope (\texttt{\#}), and a table mapping from these identifiers to their chosen pseudonyms shall be made public.

\section{The Leader}
\label{sec:leader}

\subsection{Powers}

The Leader has the power to:

\begin{enumerate}
 \item Arbitrarily modify the Code, primarily to codify basic mechanics
 \item Admit or (incl) remove any Member to or (incl) from the group
 \item Set criteria for admittance or (incl) removal from the group
 \item Establish an official language
 \item Set the number of infraction points of any Member; no fewer than negative fifty ($-50$) and no more than one hundred fifty (150).
 \item Not perform one or more of these actions
\end{enumerate}

These powers should not be used without reason. In particular, it is not recommended to change fundamental text in the Code.

\subsection{Responsibilities}

The Leader is liable to:

\begin{enumerate}
 \item Use his or her powers in order to enhance Touhou Prono
 \item Enforce the Code fairly, including on him- or herself
 \item If establishing an official language, use one widely understood or (incl) easily and quickly learnable, and translate the Code into that language
 \item Maintain records of Members and succession of Leaders
\end{enumerate}

\subsection{Resignation of Leader}

If a Leader intends to resign, then he or she must notify all Members at least sixteen (16) days in advance. During the first eight (8) days of this period, Members may apply to become the next Leader, and the current Leader must inspect at least the first sixteen (16) applications submitted, or if fewer than sixteen (16) are submitted, all that were, and choose the candidate from these applications. In case no applications are submitted, the resignation shall be deemed unsuccessful.

If an unplanned event causes a Leader to lose his or her ability to fulfill his or her position, a vacancy of sixteen (16) days is declared. During the first eight (8) days of the period, Members may apply to become the next Leader, and any Member who did not apply may read the applications and vote for one or more candidates during the following eight (8) days, in which case the candidate receiving the most votes becomes the Leader. In the event that no applications are received during the first half of the vacancy, the application period shall be extended by eight (8) days.

The criteria for such applications shall be created by the Leader at the time of his or her start and maintained at the \texttt{touhou-prono/ldrapp} repository, and may be changed by the Leader at any time except for the resignation window.

\subsection{Documentation of succession of Leaders}

The succession of Leaders, with their indices and dates of leadership, including unsuccessful resignations, must be documented.

\section{Moderators}

\subsection{Powers and responsibilities}

Moderators have the right and responsibility to:

\begin{enumerate}
 \item Enforce Chapter \ref{chapter:postreg}
 \item Maintain post infraction records
 \item Declare announcements as necessary
\end{enumerate}

\subsection{Eligibility}

No Member not satisfying the criteria below may become a Moderator:

\begin{enumerate}
 \item Have fewer than twenty-five (25) infraction points
 \item Have been a Member for at least eight (8) days
 \item Have made at least two (2) posts
\end{enumerate}

\subsection{Promotion to Moderator}

The current Leader shall select Moderators by any means complying to the Code. However, Moderators must consist of no more than one-eighth (1/8) of all Members.

\section{On holding referendums}

\subsection{General}

Changes to the Code may be proposed by any Member who had joined at least sixteen (16) days ago. All such proposals must be approved by the Leader if it is sound. If so, a vote is held among all Members who elect to participate, and the proposal becomes effective if and only if, after four (4) days of voting, there are three (3) times as many Members who voted in favor of the change than against it.

\subsection{Event wherein a proposal is declared unsound}

If a proposal is declared unsound, then the Member who submitted it may revise and resubmit it. Alternatively, in the absence of conditions defined in subsection \ref{subsec:unsound}, the Member may challenge the Leader's decision, and given no objections by at least fifteen (15) percent of non-Leader Members, rounded down to the nearest Member, the proposal shall proceed as if it had been approved.

\subsection{Inherently unsound proposals}
\label{subsec:unsound}

Rejection of such proposals may not be challenged; however, they may be revised and resubmitted as to become sound.

\begin{enumerate}
 \item Proposals in the incorrect format; see subsection \ref{subsec:propform}
 \item Proposals that do not change the Code itself; e.~g.~calling for a change in Leader
 \item Proposals that execute large amounts of (structural or otherwise) harm on the group
 \item Proposals whose changes are neither all closely related to each other nor otherwise required as a consequence for another
 \item Proposals whose identifier or (incl) name match that of another
 \item Proposals that contain many spelling or (incl) grammar errors
 \item Illegible proposals
 \item Proposals not in the official language
\end{enumerate}

\subsection{Correct format for proposals}
\label{subsec:propform}

Proposals must contain the following information:

\begin{enumerate}
 \item Identifier and pseudonym of submitter(s).
 \item Thirty-two (32) -bit identifier of the proposal.
 \item Title of proposal.
 \item Brief explanation and purpose of proposal.
 \item All necessary changes in the Code
\end{enumerate}

In addition, proposals must be typeset in \LaTeX.

\section{Limitations}

No Member may possess more than one (1) official position simultaneously.

\chapter{Regulations on posting}
\label{chapter:postreg}

\section{Infraction points}

A newly joined Member is given zero (0) infraction points, unless he or she was previously removed involuntarily, in which case he or she shall be given one hundred forty-nine (149) infraction points.

On the twenty-sixth (26th) of January, May, and September, all Members who have joined at least one hundred twenty (120) days before that date will have infraction points taken per $f : \mathbb{Z} \Rightarrow \mathbb{N}_0$, wherein $p$ is the number of infraction points previously possessed:

$$f(p) = \left\lbrace \begin{array}{rl}
 50, & p \ge 50 \\
 \min(25, p + 50), & p \le 0 \\
 \lfloor25 + .5p\rfloor, & \text{otherwise}
\end{array} \right. $$

All infractions must garner at least two (2) points.

In addition, Moderators who accumulate at least twenty-five (25) infraction points shall lose their positions.

\subsection{Minimum punishments for each infraction}
\label{subsec:postcons}

For each event wherein a Member gains infraction points, the following are the mildest punishments he or she may receive. The rightmost column is the minimum punishment permitted given received infraction points greater than or equal to the value in the first (1st) column, or (incl) total after new infraction points greater than or equal to the value in the second (2nd) column.

\begin{tabular}{|r|r|l|}
 \hline
 Received & Total & Result \\ \hline
 2 & 20 & Formal warning \\
 5 & 60 & Posting ban (3 days) \\
 7 & 80 & Posting ban (7 days) \\
 9 & 100 & Posting ban (15 days) \\
 12 & 120 & Posting ban (30 days) \\
 $\infty$ & 150 & Removal from membership \\ \hline
\end{tabular}

\subsection{Notification of punishments}

The Member receiving a punishment for posting prohibited content specified in section \ref{sec:procon} shall be notified with the following text:

\newcommand{\nintro}{You have posted content or otherwise behaved in a manner prohibited by section \ref{sec:procon} in the version of the Touhou Prono Code effective at the time this message was sent in the domain of Touhou Prono. Therefore, your post may have been edited or deleted and }
\newcommand{\nfinal}{Further violations may be punished more severely per section \ref{subsec:postcons}.}

\begin{enumerate}
 \item Formal warning: \\
 \texttt{
  \nintro you are warned to follow regulations for posting content. \nfinal
 }
 \item Posting ban: \\
 \texttt{
  \nintro you have been banned from posting for \textit{amt} days. \nfinal
 }
 \item Removal from membership: \\
  \texttt{
   \nintro you are no longer a Member of Touhou Prono.
  }
\end{enumerate}

Informal warnings are not required to be notified in a certain format.

\subsection{Transparency of infraction history}

The infraction history of all Members shall be viewable by all Moderators and the Leader.

\section{Prohibited content and behavior}
\label{sec:procon}

Posts or behavior in the domains of Touhou Prono fulfilling one or more of the following criteria may be edited to comply with regulations or deleted:

\begin{tabular}{|p{6cm}|r|r|}
 \hline
 Action & IP & IP (LA) \\ \hline
 Falsifying information about Touhou Prono or the Code, unless explicitly declared as such; impersonating another person & 2 -- 5 & 2 -- 12 \\
 Illegal content; embedded pornographic or otherwise NSFW (not safe for work) content, or links to such content without a sufficient warning & 5 -- 15 & 5 -- 150 \\
 Racial slurs; excessive profanity & 2 -- 6 & 2 -- 15 \\
 Spam or visible incompetence; intentional abuse of tags or sections; off-topic comments & 3 -- 10 & 3 -- 30 \\
 Vandalism of content, including GitHub repositories & 2 -- 10 & 2 -- 100 \\
 Abuse of position & 10 -- 80 & 10 -- 150 \\
 Promotion of any of the above (for $x$ points) & $\lfloor.5x\rfloor$ -- $2x$ & $\lfloor.5x\rfloor$ -- $4x$ \\ \hline
\end{tabular}

\section{Discouraged behavior}
\label{discon}

These behaviors are discouraged; posts doing one of the below may be edited, but do not warrant an infraction unless repeated at least three (3) times:

\begin{enumerate}
 \item Failing to highlight code, when possible
 \item Poorly formatted posts
 \item Religious or political discussions
 \item Promotion of the Python language
\end{enumerate}

\section{Explicitly permitted behavior}
\label{permcon}

The reasons below alone shall not justify editing, deleting, or punishments; however, there may be other reasons for doing so:

\begin{enumerate}
 \item Profanity, in moderation
 \item Soliciting coding help, except in the Python language
 \item Hostility
\end{enumerate}

\chapter{Language}
\label{chapter:language}

\section{The Linguistic Committee}

\subsection{General}

The Linguistic Committee (LC) consists of up to five (5) members.

\subsection{Powers}

The LC has the power to:

\begin{enumerate}
 \item Establish an official language
 \item Modify all of chapter \ref{chapter:language} with the exception of member count
\end{enumerate}

\subsection{Requirements of membership}

A Member must fulfill the following requirements in order to participate in the LC:

\begin{enumerate}
 \item Speak at least two different languages
 \item Possess a good control of the official language
 \item Have been a Member for at least thirty-two (32) days
\end{enumerate}

\subsection{Determination of membership}

The first five (5) LC members shall be appointed by the Leader.

After the first five (5), new members are admitted after one leaves, via the Artificial Language Examination (ALE; defined in section \ref{sec:ale}), which is administered for four (4) days after resignation of an LC member, or (incl) until sixteen (16) applications are received, whichever comes before the other. As many applicants to fill five positions, with the highest scores, are admitted into the LC.

\section{The Artificial Language Examination (ALE)}
\label{sec:ale}

\subsection{Production of the ALE}

A new ALE shall be produced immediately after induction of new members into the LC, in the following fashion:

\begin{enumerate}
 \item LC members collaborate with each other to develop a completely \emph{a priori} invented language.
 \item Each LC member produces ten (10) translation questions, five (5) from official to defined, and five (5) from defined to official, for a total of fifty (50).
 \item The questions and a reference to the language, plus an honor code are combined into the ALE.
\end{enumerate}

\subsection{Scoring of the ALE}

Tests are scored by each LC member, and scores from each LC member are summed for each submission.

Each question earns from zero (0) to four (4) points as specified on the scale below, for a maximum of two hundred (200) points per scorer or one thousand (1000) total.

\begin{tabular}{|r|p{5cm}|}
 \hline
 Score & Meaning \\ \hline
 4 & Completely correct; if from defined to official, is a natural translation \\
 3 & Generally correct but with minor errors, or (incl) if from defined to official, translation sounds awkward \\
 2 & Some errors in translation, especially ones reflecting lack of understanding \\
 1 & Major errors in translation, or errors reflecting severe lack of understanding \\
 0 & Completely incorrect, or blank \\
 \hline
\end{tabular}

\section{Change of official language}

An LC member may suggest a change of official language; if all members and the Leader unanimously approve of this decision, on the basis of [widespread use or (incl) ease of learning] and documentation, the following actions will be taken:

\begin{tabular}{|r|p{5cm}|}
 \hline
 Days after notification of change & Action \\ \hline
 0 & Change announced; documentation given if constructed language \\
 10 & Begin translation of Code, including language-dependent sections \\
 70 & New non-emergency changes to the Code locked until change becomes effective \\
 90 & Change becomes effective \\
 \hline
\end{tabular}

\appendix

\chapter{Reference of Members}

\verbatiminput{ml}

\chapter{Succession of Leaders}

\verbatiminput{sc}

\end{document}
